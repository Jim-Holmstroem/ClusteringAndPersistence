\documentclass{article}
\usepackage{graphicx}
\usepackage{python}
\usepackage{amsmath}
\usepackage{amssymb}
\usepackage{helpers}

\begin{document}

\title{A Solution Manual for: An Intermediate Course in Probability, Allan Gut.}
\author{
    Jim Holmstr\"{o}m\\
    Ariel Ekgren\\
    Gabriel Isheden\\
}

\maketitle

\begin{python}
import os
import re

class parseable(object):
    def __init__(self):
        raise Exception("parseable is abstract and shouldn't be instantiated")
    def __int__(self):
        return self.number 
    def __str__(self):
        return self.filename
    def __repr__(self):
        return self.filename
    def __lt__(self, other):
        return int(self) < int(other) 

class chapter(parseable):
    def __init__(self, filename):
        """
        Assumes that filename is checked to be valid
        """
        self.filename = filename
        self.number = int(filename.replace("ch",""))
        self.solutions = sorted(map(solution,
            filter(is_solution,
                filter(lambda name:not(os.path.isdir(name)),
                    os.listdir(self.filename)
                )
            )
        ))

class solution(parseable):
    def __init__(self, filename):
        """
        Assumes that filename is checked to be valid
        """
        self.filename = filename
        self.number = int(filename.replace("sol","").replace(".tex",""))

is_chapter = lambda name: bool(re.match(r'^ch[0-9]+$', name)) #$
is_solution = lambda name: bool(re.match(r'^sol[0-9]+\.tex$', name)) #$

chapters = sorted(map(chapter, 
    filter(is_chapter,
        filter(os.path.isdir, 
            os.listdir('.')
        )
    )
))

for chapter in chapters:
    print '\section{Chapter %d}'%int(chapter)
    for solution in chapter.solutions:
        filename = os.path.join(str(chapter), str(solution))
        print '\subsection{Problem %d.%d}'%(int(chapter),int(solution))
        print '\input{%s}'%filename
\end{python}

\end{document}

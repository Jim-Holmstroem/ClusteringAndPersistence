%----------------------
\emph{Metric space}
$(X, d)$ where $X$ is a finite set and $d$ is a distance between
the points in this finite set.

\begin{equation}
    |X|<\infty
\end{equation}
\begin{equation}
    d(x, x) = 0,
    d(x, y) = d(y, x),
    d(x, y) + d(y, z) \le d(x, z) \forall x, y, z \in X
\end{equation}

An intresting submetric is the \emph{ultrametric} which has a strong triangle
inequality contraint
\begin{equation}
    \max\{d(x, y), d(y, z)\} \le d(x, z) \forall x, y, z \in X
\end{equation}
also called ultrametric inequality.

Stated in another way, for an ultrametric any $3$ points, i.e. a triangle,
will have the following property
\begin{verbatim}
    <image of triangle with the sides (a, a, b <= a)>
\end{verbatim}

$\setofpartitions{X}$ is the set of partitions of $X$.
A partition of $X$
\begin{equation}
    \sigma=\set{u_i}=\disjunion\limits_i u_i \in\setofpartitions{X}
\end{equation}
where $u_i\in\sigma$ is called a block.

To be able to order partitionings we define
\begin{equation}
    \sigma \le \tau \Leftrightarrow \forall u\in\sigma\exists v\in\tau:
    u\subset v
\end{equation}

A \emph{clustering} is just a function $\Psi$ in the form of a algorithm or
procedure which maps a metric space
$(X, d)$ to a partition of $X$

\begin{equation}
    \Psi : (X, d) \rightarrow \setofpartitions{X}
\end{equation}

The function
$\Phi(X, d)\in \setofpartitions{X}$
can have $3$ important properties

\begin{description}
    \item[Scale invariant]
        Changing the scale for the distance does not change the partitioning
        \begin{equation}
            \Psi(X, d) = \Psi(X, \alpha d) \forall \alpha\in \RR_{++}
        \end{equation}
    \item[Rich]
        $\Psi(X, d)\surjective\setofpartitions{X}$ i.e. surjective or onto the
        set of partitions of $X$.
        \begin{equation}
            \forall \sigma\setofpartitions{X} \exists d : \Psi(X, d)=\sigma
        \end{equation}
    \item[Consistent]
        $d'$ is such that you decrease the intrablock distance and increase the
        extrablock distance for all blocks. This will correspond to making the
        clusters more distinct. Let $x \sim_{\Psi(X, d)} y$ denote that $x$ and
        $y$ belongs to the same block of the clustering $\Psi(X, d)$. Then $d'$
        is a transformation such that:
        \begin{equation}
            d'(x, y) : \begin{cases}
                \le d(x, y) & x\sim_{\Psi(X, d)} y\\
                \ge d(x, y) & x\not\sim_{\Psi(X, d)} y\\
            \end{cases}
        \end{equation}
        (not in notes; but it should be that $\Psi(X, d)=\Psi(X, d')\forall d'$
        fullfilling the above property)
\end{description}

%----------------------

According to the Kleinberg theorem<ref> a $\Phi$ satisfying all these properties
does not exists.

Consider the set of $3$ points $\set{a, b, c}$, then a metric can be
represented by a matrix
\begin{equation}
    \begin{tabular}{l|*{3}c}
          & a & b & c\\
        \hline
        a & 0 & x & y\\
        b & x & 0 & z\\
        c & y & z & 0\\
    \end{tabular}
\end{equation}
which satisfies the triangle inequality
\begin{equation}
    \begin{cases}
        d(a, b) + d(b, c) \le d(a, c)\\
        d(a, c) + d(c, b) \le d(a, b)\\
        d(b, a) + d(a, c) \le d(b, c)\\
    \end{cases}\Rightarrow
    \begin{cases}
        x + z \le y\\
        y + z \le x\\
        x + y \le z\\
    \end{cases}
\end{equation}

That is the distances is
\begin{equation}
    \left\{ %TODO make this a helper, setcomprehension
        (x, y, z)
    \left|
        \begin{split}
            x + z \le y\\
            y + z \le x\\
            x + y \le z\\
        \end{split}
    \right.\right\}
\end{equation}
which is an intersection of $3$ halfspaces.

%----------------------

%----------------------

%----------------------

